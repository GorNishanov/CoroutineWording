
%sdf sad
\rSec0[basic]{Basic concepts}

\setcounter{section}{6}
%\rSec1[basic.start]{Start and termination}

\rSec2[basic.start.main]{Main function}

Add underlined text to paragraph 3.

\begin{quote}
	\setcounter{Paras}{2}

\pnum
The function \tcode{main} shall not be used within
a program.
\indextext{\idxcode{main()}!implementation-defined linkage~of}%
The linkage~(\cxxref{basic.link}) of \tcode{main} is
\impldef{linkage of \tcode{main}}. A program that defines \tcode{main} as
deleted or that declares \tcode{main} to be
\tcode{inline,} \tcode{static}, or \tcode{constexpr} is ill-formed. 
\added{The function \tcode{main} shall not be a coroutine (\ref{dcl.fct.def.coroutine}).} ...
%The name \tcode{main} is
%not otherwise reserved. \enterexample member functions, classes, and
%enumerations can be called \tcode{main}, as can entities in other
%namespaces. \exitexample
\end{quote}

\setcounter{section}{7}
\setcounter{subsection}{4}
\rSec3[basic.stc.dynamic.allocation]{Allocation functions}

Modify paragraph 4 as follows:

\begin{quote}
	\setcounter{Paras}{3}
\pnum

A global allocation function is only called as the result of a new
expression~(\cxxref{expr.new}), \removed{or} called directly using the function call
syntax~(\cxxref{expr.call}), 
\added{called indirectly to allocate storage for a coroutine frame (\ref{dcl.fct.def.coroutine}),}
or called indirectly through calls to the
functions in the \Cpp standard library. \enternote In particular, a
global allocation function is not called to allocate storage for objects
with static storage duration~(\cxxref{basic.stc.static}), for objects or references
with thread storage duration~(\cxxref{basic.stc.thread}), for objects of
type \tcode{std::type_info}~(\cxxref{expr.typeid}), or for an
exception object~(\cxxref{except.throw}).
\exitnote
\end{quote}

%\setcounter{section}{6}
%\rSec1[basic.stc]{Storage duration}
%\setcounter{subsection}{4}
%%\rSec2[basic.stc.dynamic]{Dynamic storage duration}%
%\rSec3[basic.stc.dynamic.allocation]{Allocation functions}
%Add underlined text to paragraph 1.
%
%\begin{quote}
%\pnum
%\indextext{function!allocation}%
%An allocation function shall be a class member function or a global
%function; a program is ill-formed if an allocation function is declared
%in a namespace scope other than global scope or declared static in
%global scope. The return type shall be \tcode{void*}. The first
%parameter shall have type \tcode{std::size_t}~(\cxxref{support.types}). The
%first parameter shall not have an associated default
%argument~(\cxxref{dcl.fct.default}). The value of the first parameter shall
%be interpreted as the requested size of the allocation. An allocation
%function can be a function template. Such a template shall declare its
%return type and first parameter as specified above (that is, template
%parameter types shall not be used in the return type and first parameter
%type). Template allocation functions shall have two or more parameters.
%\added{An allocation function with exactly one argument is a usual allocation function.}
%\end{quote}

\begingroup
\renewcommand{\cleardoublepage}{}
\renewcommand{\clearpage}{}
\rSec0[conv]{Standard Conversions}
\endgroup
No changes are made to Clause \ref{conv} of the \Cpp Standard.
