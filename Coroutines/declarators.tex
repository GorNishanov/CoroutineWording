
%%
%% Declarators
%%
\rSec0[dcl.decl]{Declarators}

\setcounter{section}{3}
\setcounter{subsection}{4}
\rSec2[dcl.fct]{Functions}%

Add paragraph 16.

\begin{quote}
\setcounter{Paras}{15}
\pnum
A function can not be coroutine (\ref{dcl.fct.def.coroutine}) if the \grammarterm{parameter-declaration-clause} terminates with an ellipsis.
\end{quote}

\setcounter{section}{3}
\rSec1[dcl.fct.def]{Function definitions}

Add subsection \ref{dcl.fct.def.coroutine}

\setcounter{subsection}{3}
\rSec2[dcl.fct.def.coroutine]{Coroutine Functions}

%\rSec3[coroutine.definitions]{Definitions}

\pnum
A function is \defn{coroutine} if it contains
one or more \term{suspend-resume point}{}s. 

\pnum
\defn{Suspend-resume points} are introduced by \tcode{await} operator (\ref{expr.await}) in potentially-evaluated expression,
\tcode{yield} statement (\ref{stmt.yield}), 
or \tcode{for await} statement (\ref{stmt.for.await}). Every coroutine
also has an implicit initial and final \textit{suspend-resume-point} as described later in this clause. 

\pnum
\enternote
From the perspective of the caller, a coroutine is just a function with that particular signature. The fact that a function is implemented as coroutine is unobservable by the caller. 
\exitnote

\pnum
Coroutines need a set of related types and functions
to complete the definition of their semantics.
These types and functions are provided as a set of member types or typedefs
and functions in the instantiation of struct template
\tcode{coroutine_traits} (\ref{coroutine.traits}). 

\pnum
For a coroutine \tcode{f}, Let \tcode{R} be a return type and $P_1$, $P_2$, ..., $P_n$
be types of parameters. If \tcode{f} is a non-static member function then $P_1$ denotes the type of the implicit \tcode{this} parameter. 
Coroutine traits for function \tcode{f} is an instantiation of
struct template \tcode{std::experimental::coroutine_traits<R,P1,...,PN>}.
Let \tcode{F} be a \grammarterm{function-body}
\footnote{
Due to requirement of having suspend-resume points,
\grammarterm{function-body} is either a
\grammarterm{compound-statement} or 
\grammarterm{function-try-block}.
}
of that function. Then, the coroutine 
should behave as if its body were:
\begin{codeblock}
  {
     using _Tr = std::experimental::coroutine_traits<R,P1,...,PN>;
     _Tr::promise_type _Pr;
     if (_Pr.initial_suspend()) {
       suspend-resume-point // initial suspend point
     }
     try { F }
     catch(...) {
       @\textit{stop-or-propagate}@;
     }
     if (_Pr.final_suspend()) {
       suspend-resume-point // final suspend point
     }
  }
\end{codeblock}
where type alias \tcode{_Tr} and local variable \tcode{_Pr} are 
defined for
exposition only and
\textit{stop-or-propagate} is \tcode{throw} 
if \tcode{promise_type} does not have \tcode{set_exception} member function defined, and \tcode{_Pr.set_exception(std::current_exception())} otherwise. An object denoted as \tcode{_Pr} is a \defn{promise object} of
a coroutine and its type is a \defn{promise type}
of the coroutine.
%\pnum
An execution of a coroutine is suspended when it reaches a suspend-resume point.

\pnum 
A \defn{suspension} of a coroutine returns control to the current
caller of the coroutine. For the first suspend, the return value is obtained by invoking member function 
\tcode{get_return_object} (\ref{coroutine.promise})
of the \term{promise object} of the coroutine.
For the subsequent suspends, if any, the coroutine
is invoked via resumption members functions of
\tcode{coroutine_handle} (\ref{coroutine.handle}) and no return value is expected.

\pnum
A suspended coroutine can be resumed
to continue execution by invoking
resumption member functions (\ref{coroutine.handle.resumption}) of an object of \tcode{coroutine_handle<P>} type
associated with this instance of a coroutine, where type P
is a \term{promise type} of the function. 

\pnum
A coroutine may require to allocate
memory to store objects with automatic storage duration
local to the coroutine. If so, it must
use the allocator object obtained as described in 
Table~\ref{tab:coroutine.traits.requirements} in clause \ref{coroutine.traits}.

\pnum
A \defn{coroutine state} consist of 
storage for objects with automatic storage duration
that are live at the current point of execution or suspension of 
a coroutine.
\term{Coroutine state} is destroyed when
the control flows off the end of the function or
\tcode{destroy} member function (\ref{coroutine.handle.resumption}) of \tcode{coroutine_handle} object associated with that function is invoked.

\pnum
An invocation of a coroutine may incur a move operation for the parameters that may be accessed in the \grammarterm{function-body}
of coroutine after a resume. 
%These copies are defined in the same scope as a \term{promise object}. 
References to those parameters in the \grammarterm{function-body}
of the coroutine are replaced with 
references to their copies .

\pnum
If a parameter copy/move is required, class object moves are performed according to the rules described in Copying and moving class objects (\cxxref{class.copy}).

\pnum
If a parameter move, a call to \tcode{get_return_object}, or a promise object construction throws
an exception, objects with automatic storage duration (\cxxref{basic.stc.auto}) that have been
constructed are destroyed in the reverse order of their construction, any memory dynamically allocated 
for \term{coroutine state} is freed
and the search for a handler starts in the scope of the calling function. 

\pnum
A coroutine shall not have an ellipsis parameter specification. 


%\begin{codeblock}
%R f(T1 a, T2 b) {
%  // specialize coroutine_traits to discover customization points
%  using _Traits = std::experimental::coroutine_traits<R,T1,T2>;
%  <allocate memory for automatic variables>
%  <transfer parameters as needed>
%  <prepare return value>
%  if (!__pr.initial_suspend()) {
%    __resume_f(context);
%  }
%}
%
%void __resume_f(_Context* context) {
%  try {
%  } catch(...) {
%  <handle exception as requiested by the coroutine promise>
%}
%__end:  
%if (__promise.final_suspend()) {
%	<suspend-resume-point>
%	std::terminate();
%}
%<
%}
%\end{codeblock}
