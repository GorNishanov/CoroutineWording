
%%
%% Declarators
%%
\rSec0[dcl.decl]{Declarators}

\setcounter{section}{3}
\setcounter{subsection}{4}
\rSec2[dcl.fct]{Functions}%

Add paragraph 16.

\begin{quote}
\setcounter{Paras}{15}
\pnum
A function can not be resumable (\ref{dcl.fct.def.resumable}) if the \grammarterm{parameter-declaration-clause} terminates with an ellipsis.
\end{quote}

\setcounter{section}{3}
\rSec1[dcl.fct.def]{Function definitions}

Add subsection \ref{dcl.fct.def.resumable}

\setcounter{subsection}{3}
\rSec2[dcl.fct.def.resumable]{Resumable Functions}

%\rSec3[resumable.definitions]{Definitions}

\pnum
A function is \defn{resumable} if it contains
one or more \term{suspend-resume point}{}s. 

\pnum
\defn{Suspend-resume points} are introduced by \tcode{await} operator (\ref{expr.await}) in potentially-evaluated expression,
\tcode{yield} statement (\ref{stmt.yield}), 
or \tcode{for await} statement (\ref{stmt.for.await}). Every resumable function
also has an implicit initial and final \textit{suspend-resume-point} as described later in this clause. 

\pnum
Resumable functions need a set of related types and functions
to complete the definition of their semantics.
These types and functions are provided as a set of member types or typedefs
and functions in the instantiation of struct template
\tcode{resumable_traits} (\ref{resumable.traits}). 

\pnum
For a resumable function \tcode{f}, Let \tcode{R} be a return type and $P_1$, $P_2$, ..., $P_n$
be types of parameters. If \tcode{f} is a non-static member function then $P_1$ denotes the type of the implicit \tcode{this} parameter. 
Resumable traits for function \tcode{f} is an instantiation of
struct template \tcode{std::experimental::resumable_traits<R,P1,...,PN>}.
Let \tcode{F} be a \grammarterm{function-body}
\footnote{
Due to requirement of having suspend-resume points,
\grammarterm{function-body} is either a
\grammarterm{compound-statement} or 
\grammarterm{function-try-block}.
}
of that function. Then, the resumable function 
should behave as if its body were:
\begin{codeblock}
  {
     using _Tr = std::experimental::resumable_traits<R,P1,...,PN>;
     _Tr::promise_type _Pr;
     if (_Pr.initial_suspend()) {
       suspend-resume-point // initial suspend point
     }
     try { F }
     catch(...) {
       @\textit{stop-or-propagate}@;
     }
     if (_Pr.final_suspend()) {
       suspend-resume-point // final suspend point
     }
  }
\end{codeblock}
where type alias \tcode{_Tr} and local variable \tcode{_Pr} are 
defined for
exposition only and
\textit{stop-or-propagate} is \tcode{throw} 
if \tcode{promise_type} does not have \tcode{set_exception} member function defined, and \tcode{_Pr.set_exception(std::current_exception())} otherwise. An object denoted as \tcode{_Pr} is a \defn{promise object} of
a resumable function and its type is a \defn{promise type}
of the resumable function.
%\pnum
An execution of a resumable function is suspended when it reaches a suspend-resume point.

\pnum
A suspended resumable function can be resumed
to continue execution by invoking
resumption member functions (\ref{resumable.handle.resumption}) of an object of \tcode{resumable_handle<P>} type
associated with this instance of a resumable function, where type P
is a \term{promise type} of the function. 

\pnum
A resumable function may require to allocate
memory to store objects with automatic storage duration
local to the resumable function. If so, it must
use the allocator object obtained as described in 
Table~\ref{tab:resumable.traits.requirements} in clause \ref{resumable.traits}.

\pnum
A \defn{resumable function state} consist of 
storage for objects with automatic storage duration
that are live at the current point of execution or suspension of 
a resumable function.
\term{Resumable function state} is destroyed when
the control flows off the end of the function or
\tcode{destroy} member function (\ref{resumable.handle.resumption}) of \tcode{resumable_handle} object associated with that function is invoked.

\pnum 
\defn{Suspension} of a resumable function returns control to the current
caller of the resumable function. For the first suspend, the return value is obtained by invoking member function 
\tcode{get_return_object} (\ref{resumable.promise})
of the \term{promise object} of the resumable function.
For the subsequent suspends, if any, the resumable function
is invoked via resumption members functions of
\tcode{resumable_handle} (\ref{resumable.handle}) and no return value is expected.

\pnum
An invocation of a resumable function may incur a move operation for the parameters that may be accessed in the \grammarterm{function-body}
of resumable function after a resume. 
%These copies are defined in the same scope as a \term{promise object}. 
References to those parameters in the \grammarterm{function-body}
of the resumable function are replaced with 
references to their copies .

\pnum
If a parameter copy/move is required, class object moves are performed according to the rules described in Copying and moving class objects (\cxxref{class.copy}).

\pnum
If a parameter move, a call to \tcode{get_return_object}, or a promise object construction throws
an exception, objects with automatic storage duration (\cxxref{basic.stc.auto}) that have been
constructed are destroyed in the reverse order of their construction, any memory dynamically allocated 
for \term{resumable function state} is freed
and the search for a handler starts in the scope of the calling function. 

\pnum
A resumable function shall not have an ellipsis parameter specification. 


%\begin{codeblock}
%R f(T1 a, T2 b) {
%  // specialize resumable_traits to discover customization points
%  using _Traits = std::experimental::resumable_traits<R,T1,T2>;
%  <allocate memory for automatic variables>
%  <transfer parameters as needed>
%  <prepare return value>
%  if (!__pr.initial_suspend()) {
%    __resume_f(context);
%  }
%}
%
%void __resume_f(_Context* context) {
%  try {
%  } catch(...) {
%  <handle exception as requiested by the resumable promise>
%}
%__end:  
%if (__promise.final_suspend()) {
%	<suspend-resume-point>
%	std::terminate();
%}
%<
%}
%\end{codeblock}
