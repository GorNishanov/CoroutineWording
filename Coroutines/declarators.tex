
%%
%% Declarators
%%
\rSec0[dcl.decl]{Declarators}

\setcounter{section}{3}
\setcounter{subsection}{4}
%\rSec2[dcl.fct]{Functions}%

%NOTES: Fix
%Add paragraph 16.
%
%\begin{quote}
%\setcounter{Paras}{15}
%\pnum
%If the \grammarterm{parameter-declaration-clause} terminates with an ellipsis that is not part of \grammarterm{abstract-declarator}, a function shall not be coroutine (\ref{dcl.fct.def.coroutine}).
%\end{quote}

\setcounter{section}{3}
\rSec1[dcl.fct.def]{Function definitions}

\setcounter{subsection}{3}
\rSec2[dcl.fct.def.coroutine]{Coroutines}

Add this section to \ref{dcl.fct.def}.

%\rSec3[coroutine.definitions]{Definitions}
\begin{quote}
\pnum
A function is a \defn{coroutine} if it contains a \grammarterm{coroutine-return-statement} (\ref{stmt.return.coroutine}),  an \grammarterm{await-expression} (\ref{expr.await}), a \grammarterm{yield-expression} (\ref{expr.yield}), or a \grammarterm{range-based-for} (\ref{stmt.ranged}) with \tcode{co_await}.
The \grammarterm{parameter-declaration-clause} of the coroutine shall not terminate with an ellipsis that is not part of a \grammarterm{parameter-declaration}. 

%Every coroutine
%also has an implicit initial and final suspend-resume point as described later in this section. 

%NOTES: add wording for 

%\pnum
%\enternote
%From the perspective of the caller, a coroutine is just a function with that particular signature. The fact that a function is implemented as a coroutine is unobservable by the caller. 
%\exitnote

\pnum
\enterexample
\begin{codeblock}
  task<int> f();
  
  task<void> g1() {
    int i = co_await f();
    std::cout << "f() => " << i << std::endl;
  }

  template <typename... Args>
  task<void> g2(Args&&...) { // OK: ellipsis is a pack expansion
    int i = co_await f();
    std::cout << "f() => " << i << std::endl;
  }

  task<void> g3(int a, ...) { // error: variable parameter list not allowed
    int i = co_await f();
    std::cout << "f() => " << i << std::endl;
  }

\end{codeblock}
\exitexample
%
%\enterexample
%\begin{codeblock}
%  // coroutine hello world
%  generator<char> hello_fn() {
%    for (auto ch: "Hello, world") co_yield ch;
%  }
%  
%  int main() {
%    // coroutine as a lambda
%    auto hello_lambda = []()->generator<char> { 
%      for (auto ch: "Hello, world") co_yield ch; };
%    
%    for (auto ch : hello_lambda()) std::cout << ch;
%    for (auto ch : hello_fn()) std::cout << ch;
%  }
%\end{codeblock}
%\exitexample

%\pnum
%A coroutine needs a set of related types and functions
%to complete the definition of its semantics.
%These types and functions are provided as a set of member types or typedefs
%and member functions in the specializations of class template
%\tcode{std::coroutine_traits} (\ref{coroutine.traits}). 

\pnum
For a coroutine \textit{f} that is a non-static member function, let $P_1$ denote the type of the implicit object parameter (\cxxref{over.match.funcs}) and $P_2$ ... $P_n$ be the types of the function parameters; otherwise let $P_1$ ... $P_n$ be the types of the function parameters.
Let \textit{R} be the return type and \textit{F} be the \grammarterm{function-body}
%\footnote{
%Due to requirement of having suspend-resume points,
%\grammarterm{function-body} is either a
%\grammarterm{compound-statement} or 
%\grammarterm{function-try-block}.
%}
of \textit{f}, \textit{T} be the type \tcode{std::coroutine_traits<$R$,$P_1$,...,$P_n$>}, and \textit{P} be the class type denoted by \textit{T::}\tcode{promise_type}. 
%If \textit{T}\tcode{::promise_type} does not refer to a type the program is ill-formed. Type \textit{P} is the \term{promise type} of the coroutine.
Then, the coroutine behaves as if its body were:
\begin{codeblock}
  {
     @\textit{P p}@;
     co_await @\textit{p}@.initial_suspend(); // initial suspend point
     @\textit{F'}@
  @\textit{final_suspend}@:
     co_await @\textit{p}@.final_suspend(); // final suspend point
  }
\end{codeblock}
where \textit{F'} is 
\begin{codeblock}
	try {@\textit{ F }@} catch(...) { @\textit{p}@.set_exception(std::current_exception()); }
\end{codeblock}
if the \grammarterm{unqualified-id} \tcode{set_exception} is found in the scope of \textit{P}
as if by class member access lookup (\cxxref{basic.lookup.classref}), and \textit{F'} is \textit{F} otherwise.
An object denoted as \textit{p} is the \defn{promise object} of
the coroutine and type $P$ is the \defn{promise type}
of the coroutine.

%\pnum
%An execution of a coroutine is suspended when it reaches a suspend-resume-point. 
%A suspension of a coroutine returns control to the current
%caller of the coroutine. For the first return of control from the coroutine, the return value is obtained via
%expression \tcode{$p$.get_return_object()}.

\pnum
When a coroutine returns to its caller, the return value is obtained by a call to \tcode{$p$.get_return_object()}. A call to a \tcode{get_return_object} is sequenced before the call to \tcode{initial_suspend}.

%\enternote
%For subsequent suspends, if any, the coroutine
%is invoked via resumption member functions of 
%%NOTE massage
%\tcode{std::coroutine_handle} (\ref{coroutine.handle}) and no return value is expected.
%\exitnote

\pnum
A suspended coroutine can be resumed
to continue execution by invoking
a resumption member function (\ref{coroutine.handle.resumption}) of an object of type \tcode{coroutine_handle<$P$>} 
associated with this instance of the coroutine. The function that invoked a resumption member function is called \term{resumer}. Invoking a resumption member function for a coroutine that is not suspended results in undefined behavior. 

\pnum
A \defn{coroutine state} consists of 
storage for objects with automatic storage duration of a coroutine.
An implementation may need to allocate
memory for the coroutine state. If so, it shall obtain the storage by calling an
\term{allocation function}~(\cxxref{basic.stc.dynamic.allocation}).
The allocation function's name is looked up in the scope of the promise type of the coroutine. If this lookup fails to find the name, the allocation function's name is looked up in the global scope. If the lookup finds an allocation function that takes exactly one parameter, it will be used, otherwise, all parameters of the coroutine are passed to the allocation function after the size parameter in order.

\pnum
The coroutine state is destroyed when
the control flows off the end of the coroutine or
the \tcode{destroy} member function (\ref{coroutine.handle.resumption}) of an object of \tcode{std::coroutine_handle<\textit{P}}> associated with this coroutine is invoked. In the latter case objects with automatic storage duration that are in scope
at the suspend point are destroyed in the reverse order of the construction. If the coroutine state required dynamic allocation, the storage is released by calling a deallocation
function~(\cxxref{basic.stc.dynamic.deallocation}). If \tcode{destroy} is called for a coroutine that is not suspended, the program has undefined behavior.

\pnum
The deallocation function's name is looked up in the scope of the coroutine promise type. If this lookup fails to find the name, the deallocation function's name is looked up in the global scope. If deallocation function lookup finds both a usual deallocation function with only a pointer parameter and a usual deallocation function with both a pointer parameter and a size parameter, then the selected deallocation function shall be the one with two parameters. Otherwise, the selected deallocation function shall be the function with one parameter. 

%introduce metavariable instead of talking about
%coroutine_handle<>

\pnum
When a coroutine is invoked, a copy of each parameter is created in the coroutine state. Each such copy is direct-initialized from an lvalue referring to the corresponding parameter if it is an lvalue reference, and an xvalue referring to it otherwise. 
A reference to a parameter in the function-body of the coroutine is replaced by a reference to the copy of the parameter.
Creation of a copy of a parameter can be omitted by continuing to refer to the corresponding coroutine parameter if the meaning of the program will be unchanged except for the execution of constructors and destructors involved in creation of a parameter copy.
  %The copy/move operations are indeterminately sequenced with respect to each other.

%\pnum
%An invocation of a coroutine may incur an extra copy/move operation for the parameters.
%%These copies are defined in the same scope as a \term{promise object}. 
%A references to a parameter in the \grammarterm{function-body}
%of the coroutine is replaced by a 
%reference to the copy of the parameter.
%If a parameter copy/move is required, class object moves are performed according to the rules described in \cxxref{class.copy}/28 [class.copy].
%
%\enternote
%This transformation could look as follows:
%
%\begin{codeblock}
%  auto foo(A a, B& b, C&& c, D* d) {
%     A   a' = move(a);
%     B&  b' = b;
%     C&& c' = move(c);
%     D*  d' = d;
%     yield 5; // lifetime of parameters end as per 5.2.2/[expr.call]
%     ...// any parameter mentioned here will refer to its copy
%  } 
%\end{codeblock}
%\exitnote

%\ednote{This is what we currently say about parameters}

%NOTE: massage
%\pnum
%The region of a coroutine state storing parameters to the coroutine is copy-initialized with the parameter values.
%\enternote 
%This copy may be elided (\cxxref{class.copy}).
%\exitnote

%NOTE: say something that we need to check
%Will be moved to coroutine state and can elide the move.
%syntactic constraint will be checked.

%NOTE: eventual-return type not defined anywhere

%\pnum
%If during the coroutine state initialization, a call to \tcode{get_return_object}, or a promise object construction throws
%an exception, objects with automatic storage duration (\cxxref{basic.stc.auto}) that have been
%constructed are destroyed in the reverse order of their construction, 
%any memory dynamically allocated 
%for the coroutine state is freed and the search for a handler starts in the scope of the calling function. 

\pnum
If the unqualified-id \tcode{get_return_object_on_allocation_failure} is looked up in the scope of class \textit{P}
as if by class member access lookup (\cxxref{basic.lookup.classref}), and if a declaration was found,
%If type \textit{P} defines static member function \tcode{get_return_object_on_allocation_failure} (\ref{coroutine.traits}), 
 then \tcode{std::nothrow_t} forms of allocation and deallocation functions will be used. If an allocation function returned \tcode{nullptr}, the coroutine must return control to the caller of the coroutine and the return value shall be obtained by a call to \tcode{P::get_return_object_on_allocation_failure()}.

\enterexample
\begin{codeblock}
// using nothrow operator new
struct generator {
  using handle = std::coroutine_handle<promise_type>;
  struct promise_type {
    int current_value;
    static auto get_return_object_on_allocation_failure() { return generator{nullptr}; }
    auto get_return_object() { return generator{handle::from_promise(*this)}; }
    auto initial_suspend() { return std::suspend_always{}; }
    auto final_suspend() { return std::suspend_always{}; }
    auto yield_value(int value) { 
      current_value = value; 
      return std::suspend_always{};
    }
  };
  bool move_next() { return coro ? (coro.resume(), !coro.done()) : false; }
  int current_value() { return coro.promise().current_value; }
  ~generator() { if(coro) coro.destroy(); }
private:
  generator(handle h) : coro(h) {}
  handle coro;
};
generator f() { co_yield 1;  co_yield 2; }
 
int main() {
  auto g = f();
  while (g.move_next()) std::cout << g.current_value() << std::endl;
}

\end{codeblock}
\exitexample

\pnum
\enterexample
\begin{codeblock}
  // using a stateful allocator
  class Arena;
  struct my_coroutine {
    struct promise_type {
      ...
      template <typename... TheRest>
      void* operator new(std::size_t size, Arena& pool, TheRest const&...) {
        return pool.allocate(size);
      }
      void* operator delete(void* p, std::size_t size) {
      	// reference to a pool is not available
      	// to the delete operator and should be stored
      	// by the allocator as a part of the allocation
        return Arena::deallocate(p, size);
      }
    };
  };
  
  my_coroutine (Arena& a) {
    // will call my_coroutine::promise_type::operator new(<required-size>, a)
    // to obtain storage for the coroutine state
    co_yield 1;
  }
  
  int main() {
    Pool memPool;
    for (int i = 0; i < 1'000'000; ++i) my_coroutine(memPool);
  };
\end{codeblock}
\exitexample
\end{quote}