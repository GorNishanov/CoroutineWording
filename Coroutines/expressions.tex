

\setcounter{chapter}{4}
\rSec0[expr]{Expressions}

\setcounter{section}{2}
\rSec1[expr.unary]{Unary expressions}

In this section add the \tcode{await} \grammarterm{cast-expression} 
to the rule for \grammarterm{unary-expression}.

\begin{bnf}
	\nontermdef{unary-expression}\br
	postfix-expression\br
	\terminal{++} cast-expression\br
	\terminal{-{-}} cast-expression\br
	\added{\terminal{await} cast-expression}\br
	unary-operator cast-expression\br
	\terminal{sizeof} unary-expression\br
	\terminal{sizeof (} type-id \terminal{)}\br
	\terminal{sizeof ...} \terminal{(} identifier \terminal{)}\br
	\terminal{alignof (} type-id \terminal{)}\br
	noexcept-expression\br
	new-expression\br
	delete-expression\br
\end{bnf}

Add subsection 5.3.9.

\setcounter{subsection}{8}
\rSec2[expr.await]{Await}

\pnum
The \tcode{await} operator is used to suspend evaluation of the enclosing resumable function (\ref{dcl.fct.def.resumable}) while awaiting
for completion of the computation represented by the operand expression.

\pnum
The presence of an \tcode{await} operator in a potentially-evaluated expression makes the enclosing function a resumable function.

\pnum
An \tcode{await} expression shall not appear in a potentially-evaluated expression in a catch clause of a try block.

\pnum
An \tcode{await} expression shall not appear in an \grammarterm{initializer-clause} of a default parameter (\cxxref{dcl.fct.default}).

%\rSec2[expr.await.evaluation]{Evaluation of await operator}

\pnum
An \tcode{await} expression of the form

\begin{ncbnf}
	\terminal{await} cast-expression
\end{ncbnf}

is equivalent to \footnote{if it were possible to write
an expression in terms of a block, where return from the 
block becomes the result of the expression}

\begin{codeblock}
{
  auto && __expr = @\grammarterm{cast-expression}@;
  if ( !await-ready-expr ) {
    await-suspend-expr;
    @\textit{suspend-resume-point}@
  }
  return await-resume-expr;
}
\end{codeblock}

if the type of \textit{await-suspend-expr} is \tcode{void}, otherwise it is equivalent to

\begin{codeblock}
{
  auto && __expr = @\grammarterm{cast-expression}@;
  if ( !await-ready-expr && await-suspend-expr ) {
    @\textit{suspend-resume-point}@
  }
  return await-resume-expr;
}
\end{codeblock}

where \tcode{__expr} is a variable defined for
exposition only, and \tcode{_ExprT} is the type of the
\grammarterm{cast-expression}, and \tcode{_ResumableHandle}
is an object of the \tcode{resumable_handle} type specialized for an enclosing function,
and \textit{await-ready-expr}, \textit{await-suspend-expr}, and \textit{await-expr} are
determined as follows:

\begin{itemize}
	\item if \tcode{_ExprT} is a class type, the \grammarterm{unqualified-id}{s}
	\tcode{await_ready}, \tcode{await_suspend} and \tcode{await_resume} are 
	looked up in the scope of class \tcode{\mbox{_ExprT}}
	as if by class member access lookup~(\cxxref{basic.lookup.classref}), and if it finds at least one declaration, 
	\tcode{await_ready}, \tcode{await_suspend}, and \tcode{await_resume} are
	\tcode{__expr.await_ready()}, \tcode{__expr.await_suspend(_ResumableHandle)} and \tcode{__expr.await_resume()},
	respectively;
	
	\item otherwise, \textit{await-ready}, \textit{await-suspend} and \textit{await-resume} are 
	\tcode{await_ready(__expr)}, \tcode{await_suspend(__expr, _ResumableHandle)}, and \tcode{await_resume(__expr)} 
	respectively, where 
	\tcode{await_ready}, \tcode{await_suspend}, and \tcode{await_resume} are 
	looked up in the associated namespaces~(\cxxref{basic.lookup.argdep}).
	\enternote Ordinary unqualified lookup~(\cxxref{basic.lookup.unqual}) is not
	performed. \exitnote
\end{itemize}

%\enterexample
%\begin{codeblock}
%	int result = await async([]{return 5;});
%\end{codeblock}
%\exitexample%

\pnum
An \tcode{await} expression may appear as an unevaluated operand (\cxxref{expr.typeid}, \cxxref{expr.sizeof}, \cxxref{expr.unary.noexcept}, \cxxref{dcl.type.simple}). The presence of such an \tcode{await} expression does not make the enclosing function resumable and can be used to examine the type of an \tcode{await} expression.
 
\enterexample
\begin{codeblock}	
std::future<int> f() noexcept;
	
int main() {
  using t = decltype(await f()); // t is int
  static_assert(sizeof(await f()) == sizeof(int));
  cout << typeid(await f()).name() << endl;
  cout << noexcept(await f()) << endl;
}
\end{codeblock}
\exitexample%

\pnum
An \tcode{await} expression may only appear in a resumable function 
with an eventual return type, i.e a resumable function shall have the \tcode{set_result} member
function defined in its \term{promise type} (\ref{dcl.fct.def.resumable}).
