
\rSec0[intro.scope]{Scope}

\pnum
This document describes extensions to the C++ 
Programming Language (Clause \ref{intro.refs}) that
enable definition of coroutines. These extensions include 
new syntactic forms and modifications to existing language semantics.

\pnum
The International Standard, ISO/IEC 14882:2017, provides important context
and specification for this document. This document is 
written as a set of changes against that specification. Instructions
to modify or add paragraphs are written as explicit instructions. 
Modifications made directly to existing text from the International
Standard use \added{underlining} to represent added text and
\removed{strikethrough} to represent deleted text. 

\begingroup
\renewcommand{\cleardoublepage}{}
\renewcommand{\clearpage}{}
\rSec0[intro.refs]{Normative references}
\endgroup

\pnum
The following documents are referred to in the text in such a way that some or all of their content constitutes requirements of this document. For dated references, only the edition cited applies. For undated references, the latest edition of the referenced document (including any amendments) applies.

\begin{itemize}
	\item ISO/IEC 14882:2017, \doccite{Programming Languages -- \Cpp}
\end{itemize}

ISO/IEC 14882:2017 is hereafter called the \defn{\Cpp Standard}.
%
Beginning with Clause 5, all clause and subclause numbers, titles,
and symbolic references in [brackets] refer to the corresponding elements of the \Cpp Standard. Clauses 1 through 4 of this document
%are introductory material and 
are unrelated to the similarly-numbered clauses and subclauses of the \Cpp Standard.

% NOTES: N4302 use transaction memory edits to fix this. 

\begingroup
\renewcommand{\cleardoublepage}{}
\renewcommand{\clearpage}{}
\rSec0[intro.defs]{Terms and definitions}
\endgroup

No terms and definitions are listed in this document.
ISO and IEC maintain terminological databases for use in standardization at the following addresses:

\begin{itemize}
	\item ISO Online browsing platform: available at \url{http://www.iso.org/obp}
	\item IEC Electropedia: available at \url{http://www.electropedia.org/}
\end{itemize}
%\pnum
%This revision updates N4499 with changes proposed in papers P0054 and P0070.

\rSec0[intro]{General}

%\rSec1[intro.ack]{Acknowledgements}

%This work is the result of a collaboration of researchers in industry and academia. We wish to thank people who made valuable contributions within and outside these groups, including
%Artur Laksberg,
%Chandler Carruth,
%David Vandevoorde,
%Deon Brewis,
%Eric Fiselier,
%Gabriel Dos Reis,
%Herb Sutter,
%James McNellis,
%Jens Maurer,
%Jonathan Caves,
%Lawrence Crowl,
%Lewis Baker,
%Michael Wong,
%Nick Maliwacki,
%Niklas Gustafsson,
%Pablo Halpern,
%Richard Smith,
%Robert Schumacher,
%Shahms King,
%Slava Kuznetsov,
%Stephan T. Lavavej,
%Tongari J,
%Vladimir Petter,
%and many others not named here who contributed to
%the discussion.

%
%Add the definitions of ``suspend-resume-point'' and ``coroutine''.
%
%\setcounter{subsection}{26}
%\begin{quote}
%\indexdefn{suspend-resume-point}%
%\definition{suspend-resume-point}{defns.suspend.resume}
%A point in a \grammarterm{function-body}
%where evaluation of a function can be suspended
%with possibility of resuming it later
%via a call to a member function of a
%\tcode{coroutine_handle} object associated with the suspended function.
%\end{quote}
%
%\begin{quote}
%	\indexdefn{coroutine}%
%	\definition{coroutine}{defns.coroutine.function}
%	A function defined with \grammarterm{function-body} that
%	contains one or more suspend-resume-points.
%\end{quote}

%%
%% Implementation compliance
%%
\rSec1[intro.compliance]{Implementation compliance}

%\pnum
Conformance requirements for this specification shall be the same as those 
defined in subclause 1.4 of the \Cpp Standard. 
\enternote 
Conformance is defined
in terms of the behavior of programs.
\exitnote

%%
%% Feature-testing recommendations
%%
\rSec1[intro.features]{Feature testing}

An implementation that provides support for this document shall define the feature test macro in Table~\ref{tab:info.features}.

\begin{floattable}{Feature-test macro}{tab:info.features}
{lll}
\topline
\lhdr{Name} & \chdr{Value} & \rhdr{Header} \\
\capsep
\tcode{__cpp_coroutines}  & \tcode{201707} & \textit{predeclared}      \\
\end{floattable}

%\rSec1[intro.ack]{Acknowledgments}
%
%\pnum
%The design of this specification is based, in part, on a concept 
%specification of the algorithms part of the C++ standard library, known 
%as ``The Palo Alto'' report (WG21 N3351), which was developed by a large 
%group of experts as a test of the expressive power of the idea of 
%concepts. Despite syntactic differences between the notation of the 
%Palo Alto report and this document, the report can be seen as a 
%large-scale test of the expressiveness of this document.

\rSec1[intro.execution]{Program execution}
In subclause \cxxref{intro.execution} of the \Cpp Standard modify paragraph 7 to read:
\begin{quote}
\setcounter{Paras}{6}
\pnum 
An instance of each object with automatic storage 
duration~(\cxxref{basic.stc.auto}) is associated with each entry into its 
block. Such an object exists and retains its last-stored value during 
the execution of the block and while the block is suspended (by a call 
of a function\added{, suspension of a coroutine (\ref{expr.await})}, 
or receipt of a signal). 
\end{quote}

\rSec1[lex]{Lexical conventions}

In subclause \cxxref{lex.key} of the \Cpp Standard 
add the keywords \tcode{co_await}, \tcode{co_yield}, and
\tcode{co_return} to Table~4 "Keywords".


%sdf sad
%\setcounter{chapter}{2}
\rSec1[basic]{Basic concepts}

%\setcounter{section}{6}
%\rSec1[basic.start]{Start and termination}

%\rSec2[basic.start.main]{Main function}

In subclause \cxxref{basic.start.main} of the \Cpp Standard 
add underlined text to paragraph 3.

\begin{quote}
	\setcounter{Paras}{2}
	
	\pnum
	The function \tcode{main} shall not be used within
	a program.
	\indextext{\idxcode{main()}!implementation-defined linkage~of}%
	The linkage~(\cxxref{basic.link}) of \tcode{main} is
	\impldef{linkage of \tcode{main}}. A program that defines \tcode{main} as
	deleted or that declares \tcode{main} to be
	\tcode{inline,} \tcode{static}, or \tcode{constexpr} is ill-formed. 
	\added{The function \tcode{main} shall not be a coroutine (\ref{dcl.fct.def.coroutine}).}
	The name \tcode{main} is
	not otherwise reserved. \enterexample member functions, classes, and
	enumerations can be called \tcode{main}, as can entities in other
	namespaces. \exitexample
\end{quote}

\rSec1[basic.stc.dynamic]{Dynamic storage duration}

%\setcounter{section}{7}
%\setcounter{subsection}{4}
%\rSec3[basic.stc.dynamic.allocation]{Allocation functions}

In subclause \cxxref{basic.stc.dynamic.allocation} of the \Cpp Standard 
modify paragraph 4 as follows:

\begin{quote}
	\setcounter{Paras}{3}
	\pnum
	A global allocation function is only called as the result of a new
	expression~(\cxxref{expr.new}), \removed{or} called directly using the function call
	syntax~(\cxxref{expr.call}), 
	\added{called indirectly to allocate storage for a coroutine frame (\ref{dcl.fct.def.coroutine}),}
	or called indirectly through calls to the
	functions in the \Cpp standard library. \enternote In particular, a
	global allocation function is not called to allocate storage for objects
	with static storage duration~(\cxxref{basic.stc.static}), for objects or references
	with thread storage duration~(\cxxref{basic.stc.thread}), for objects of
	type \tcode{std::type_info}~(\cxxref{expr.typeid}), or for an
	exception object~(\cxxref{except.throw}).
	\exitnote
\end{quote}
