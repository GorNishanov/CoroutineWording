
%%
%% Overloading
%%
\setcounter{chapter}{12}
\rSec0[over]{Overloading}

\setcounter{section}{4}
\rSec1[over.oper]{Overloaded operators}

Add \tcode{co_await} to the list of operators in paragraph 1 before operators \tcode{()} and \tcode{[]}.

\setcounter{section}{3}
\setcounter{subsection}{1}
\setcounter{subsubsection}{1}
\rSec3[over.match.oper]{Operators in expressions}%

Change \ref{over.match.oper}/9:

\begin{quote}
\setcounter{Paras}{8}
\pnum
If the operator is the operator
\tcode{,},
the unary operator
\tcode{\&},
\removed{or} the operator
\tcode{->},
\added{or the operator \tcode{co_await}},
and there are no viable functions, then the operator is
assumed to be the built-in operator and interpreted according to
Clause~\ref{expr}.
\end{quote}

Add a new paragraph after paragraph 8:

\begin{quote}
\setcounter{Paras}{8}
\pnum
  When operator \tcode{co_await} returns, the \tcode{co_await} operator is applied to the value returned. The resulting \tcode{co_await} operator is assumed to be the built-in operator and interpreted according to Clause~\ref{expr}.
\end{quote}

%\setcounter{subsection}{8}
%\rSec2[over.await]{Await operator}%
%
%\pnum
%If there is no user-declared \tcode{operator await} for type \tcode{X}, but there is a declared member with a name \tcode{await_suspend}, \tcode{await_ready}, or \tcode{await_resume}, then the implementation shall provide the implicit definition of \tcode{operator await} in such a way that the result of the evaluation of an implicit \tcode{operator await(\textit{v})} is \textit{v} itself.

