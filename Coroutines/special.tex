
\setcounter{chapter}{11}
\begingroup
\renewcommand{\cleardoublepage}{}
\renewcommand{\clearpage}{}
\rSec0[special]{Special member functions}
\endgroup

\setcounter{section}{0}
\rSec1[class.ctor]{Constructors}%

Add new paragraph after paragraph 5.

	\setcounter{Paras}{5}
\begin{quote}
	\pnum A constructor shall not be a coroutine.
\end{quote}

\setcounter{section}{3}
\rSec1[class.dtor]{Destructors}%

Add new paragraph after paragraph 16.

	\setcounter{Paras}{16}
\begin{quote}
	\pnum A destructor shall not be a coroutine.
\end{quote}

\setcounter{section}{7}
\rSec1[class.copy]{Copying and moving class objects}%

Add a bullet to paragraph 31:

\begin{quote}
\begin{itemize}
	\item in a coroutine (\ref{dcl.fct.def.coroutine}), a copy of a coroutine parameter  can be omitted
	and references to that copy replaced with references to the corresponding parameter if the meaning of the program will
	be unchanged except for the execution of a constructor and destructor for the parameter copy object
\end{itemize}
\end{quote} 

Modify paragraph 33 as follows:

\begin{quote}
\setcounter{Paras}{32}
\pnum
When the criteria for elision of a copy/move operation are met,
but not for an \nonterminal{exception-declaration},
and the object
to be copied is designated by an lvalue,
or when the \grammarterm{expression} in a \tcode{return} \added{or } \tcode{\added{co_return}} statement
is a (possibly parenthesized) \grammarterm{id-expression}
that names an object with automatic storage duration declared in the body
or \grammarterm{parameter-declaration-clause} of the innermost enclosing
function or \grammarterm{lambda-expression},
overload resolution to select the constructor
for the copy \added{or the }\tcode{\added{return_value}}\added{ overload to call} is first performed as if the object were designated by an rvalue.
If the first overload resolution fails or was not performed,
or if the type of the first parameter of the selected
constructor \added{or }\tcode{\added{return_value}}\added{ overload }is not an rvalue reference to the object's type (possibly cv-qualified),
overload resolution is performed again, considering the object as an lvalue.
\enternote This two-stage overload resolution must be performed regardless
of whether copy elision will occur. It determines the constructor\added{ or }\tcode{\added{return_value}}\added{ overload } to be called if
elision is not performed, and the selected constructor\added{ or }\tcode{\added{return_value}}\added{ overload } must be accessible even if
the call is elided. \exitnote
\end{quote}