
\setcounter{chapter}{11}
\rSec0[special]{Special member functions}

In this section add new paragraph after paragraph 5.

\begin{quote}
	\setcounter{Paras}{5}
	\pnum
	A special member function shall not be a coroutine.
\end{quote}

\setcounter{section}{7}
\rSec1[class.copy]{Copying and moving class objects}%

% Jens: 8.4.4p10: remove "considered as xvalue", make "elided" a note with xref to 12.8, add elision condition to 12.8, expand special "return" case in 12.8 to try xvalue (= move) first, then lvalue. 

Add underlined text to paragraph 31.

\begin{quote}
\setcounter{Paras}{30}
\pnum
\indextext{temporary!elimination~of}%
\indextext{elision!copy constructor|see{constructor, copy, elision}}%
\indextext{elision!move constructor|see{constructor, move, elision}}%
\indextext{constructor!copy!elision}%
\indextext{constructor!move!elision}%
When certain criteria are met, an implementation is
allowed to omit the copy/move construction of a class object,
even if the constructor selected for the copy/move operation and/or the
destructor for the object have
\indextext{side effects}%
side effects.  In such cases, the
implementation treats the source and target of the
omitted copy/move operation as simply two different ways of
referring to the same object, and the destruction of
that object occurs at the later of the times when the
two objects would have been destroyed without the
optimization.\footnote{Because only one object is destroyed instead of two,
  and one copy/move constructor
  is not executed, there is still one object destroyed for each one constructed.}
This elision of copy/move operations, called
\indexdefn{copy elision|see{constructor, copy, elision; constructor, move, elision}}%
\indexdefn{elision!copy|see{constructor, copy, elision; constructor, move, elision}}%
\indexdefn{constructor!copy!elision}\indexdefn{constructor!move!elision}\term{copy elision},
is permitted in the
following circumstances (which may be combined to
eliminate multiple copies):

\begin{itemize}
  \item in a \tcode{return} statement in a function with a class return type,
  when the expression is the name of a non-volatile
  automatic object
  (other than a function or catch-clause parameter)
  with the same cv-unqualified type as
  the function return type, the copy/move operation can be
  omitted by constructing the automatic object directly
  into the function's return value
  \item
  When a parameter would be copied/moved to the coroutine state (\ref{dcl.fct.def.coroutine}) copy move can be omitted by continuing to refer to the function parameters instead of referring to their copies in the coroutine state.
\end{itemize}
\end{quote}