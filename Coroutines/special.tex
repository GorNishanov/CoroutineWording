
\begingroup
\renewcommand{\cleardoublepage}{}
\renewcommand{\clearpage}{}
\rSec0[special]{Special member functions}
\endgroup

\setcounter{section}{0}
\rSec1[class.ctor]{Constructors}%

Add new paragraph after paragraph 10.

	\setcounter{Paras}{10}
\begin{quote}
	\pnum A constructor shall not be a coroutine.
\end{quote}

\setcounter{section}{3}
\rSec1[class.dtor]{Destructors}%

Add new paragraph after paragraph 16.

	\setcounter{Paras}{16}
\begin{quote}
	\pnum A destructor shall not be a coroutine.
\end{quote}

\pagebreak

\setcounter{section}{7}
\rSec1[class.copy]{Copying and moving class objects}%
\setcounter{subsection}{2}
\rSec2[class.copy.elision]{Copy/move elision}%

Add a bullet to paragraph 1:

\begin{quote}
\begin{itemize}
	\item in a coroutine (\ref{dcl.fct.def.coroutine}), a copy of a coroutine parameter  can be omitted
	and references to that copy replaced with references to the corresponding parameter if the meaning of the program will
	be unchanged except for the execution of a constructor and destructor for the parameter copy object
\end{itemize}
\end{quote} 

Modify paragraph 3 as follows:

\begin{quote}
\setcounter{Paras}{2}
\pnum
In the following copy-initialization contexts, a move operation might be used instead of a copy operation:
\begin{itemize}
\item If the \grammarterm{expression} in a \tcode{return} \added{or } \tcode{\added{co_return}} statement \iref{stmt.return}
is a (possibly parenthesized) \grammarterm{id-expression}
that names an object with automatic storage duration declared in the body
or \grammarterm{parameter-declaration-clause} of the innermost enclosing
function or \grammarterm{lambda-expression}, or

\item if the operand of a \grammarterm{throw-expression}\cxxref{expr.throw}
is the name of a non-volatile automatic object
(other than a function or catch-clause parameter)
whose scope does not extend beyond the end of the innermost enclosing
\grammarterm{try-block} (if there is one),
\end{itemize}
overload resolution to select the constructor
for the copy \added{or the }\tcode{\added{return_value}}\added{ overload to call} is first performed as if the object were designated by an
rvalue.
If the first overload resolution fails or was not performed,
or if the type of the first parameter of the selected
constructor \added{or }\tcode{\added{return_value}}\added{ overload }is not an rvalue reference to the object's type (possibly cv-qualified),
overload resolution is performed again, considering the object as an lvalue.
\begin{note}
This two-stage overload resolution must be performed regardless
of whether copy elision will occur. It determines the constructor \added{or }\tcode{\added{return_value}}\added{ overload }to be called if
elision is not performed, and the selected constructor \added{or }\tcode{\added{return_value}}\added{ overload }must be accessible even if
the call is elided.
\end{note}

\end{quote}
