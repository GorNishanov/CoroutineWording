\begingroup
\renewcommand{\cleardoublepage}{}
\renewcommand{\clearpage}{}
\rSec0[stmt.stmt]{Statements}%
\endgroup

\setcounter{section}{4}
\rSec1[stmt.iter]{Iteration statements}%
Add the underlined text to paragraph 1.
%NOTE: change the grammar

\begin{quote}
\pnum
Iteration statements specify looping.

\indextext{statement!\idxcode{while}}%
\indextext{statement!\idxcode{do}}%
\indextext{statement!\idxcode{for}}%
%
\begin{bnf}
	\nontermdef{iteration-statement}\br
	\terminal{while (} condition \terminal{)} statement\br
	\terminal{do} statement \terminal{while (} expression \terminal{) ;}\br
	\terminal{for (} for-init-statement condition\opt \terminal{;} expression\opt \terminal{)} statement\br
	\terminal{for} \terminal{\added{co_await\opt{}}} 
    \terminal{(} for-range-declaration \terminal{:} for-range-initializer \terminal{)} statement\br
%	\added{\terminal{cofor (} for-range-declaration \terminal{:} for-range-initializer \terminal{)} statement}
\end{bnf}
\end{quote}

% goes inside and need fixing goes after

\setcounter{subsection}{3}
\rSec2[stmt.ranged]{The range-based \tcode{for} statement}%
\indextext{statement!range~based \idxcode{for}}

Add the underlined text to paragraph 1.

\begin{quote}
\setcounter{Paras}{0}
\pnum
For a range-based \tcode{for} statement of the form

\begin{ncbnf}
  \terminal{for} \terminal{\added{co_await\opt{}}} \terminal{(} for-range-declaration : expression \terminal{)} statement
\end{ncbnf}

let \textit{range-init} be equivalent to the \grammarterm{expression} surrounded
by parentheses\footnote{this ensures that a top-level comma operator cannot be
  reinterpreted as a delimiter between \grammarterm{init-declarator}{s} in the
  declaration of \tcode{__range}.}

\begin{ncbnf}
  \terminal{(} expression \terminal{)}
\end{ncbnf}

and for a range-based \tcode{for} statement of the form

\begin{ncbnf}
  \terminal{for} \terminal{\added{co_await\opt{}}} \terminal{(} for-range-declaration \terminal{:} braced-init-list \terminal{)} statement
\end{ncbnf}

let \textit{range-init} be equivalent to the \grammarterm{braced-init-list}. In each case, a
range-based \tcode{for} statement
is equivalent to

\begin{codeblock}
  {
    auto && __range = range-init;
    for ( auto __begin =  @\added{\tcode{co_await}\opt{}}@ begin-expr,
    __end = end-expr;
    __begin != __end;
     @\added{\tcode{co_await}\opt{}}@ ++__begin ) {
      @\textit{for-range-declaration}@ = *__begin;
      @\textit{statement}@
    }
  }
\end{codeblock}

where 
\tcode{\added{co_await}}\added{ is present if and only if it appears immediately after the \tcode{for} keyword, and}\linebreak
\tcode{__range}, \tcode{__begin}, and \tcode{__end} are variables defined for
exposition only, and \tcode{_RangeT} is the type of the
\grammarterm{}{expression}, and \textit{begin-expr} and \textit{end-expr} are
determined as follows: ...

%\ednote{The remainder of paragraph 1 remains unchanged and is not included here.}

\end{quote}


Add the following paragraph after paragraph 2.

\begin{quote}
	\setcounter{Paras}{2}
	\pnum
	A range-based \tcode{for} statement with \tcode{co_await} shall appear only within a suspension context of a function (\ref{expr.await}).
\end{quote}
%\setcounter{subsection}{4}
%\rSec2[stmt.for.await]{The \tcode{cofor} statement}%
%
%Add this section to \ref{stmt.iter}.
%
%
%\begin{quote}
%\pnum
%A \tcode{cofor} statement of the form
%
%\begin{ncbnf}
%	\terminal{cofor (} for-range-declaration : expression \terminal{)} statement
%\end{ncbnf}
%is equivalent to
%
%\begin{codeblock}
%	{
%		auto && __range = range-init;
%		for ( auto __begin = co_await begin-expr,
%	+	__end = end-expr;
%		__begin != __end;
%		co_await ++__begin ) {
%			@\textit{for-range-declaration}@ = *__begin;
%			@\textit{statement}@
%		}
%	}
%\end{codeblock}
%
%where \tcode{__range}, \tcode{__begin}, \tcode{__end}, 
%\textit{range-init}, \textit{begin-expr}, and \textit{end-expr} are defined as in the range-based \tcode{for} statement (\cxxref{stmt.ranged}).
%
%\end{quote}

\setcounter{section}{5}
\rSec1[stmt.jump]{Jump statements}%

In paragraph 1 add two productions to the grammar:

\begin{quote}
  \begin{bnf}
    \nontermdef{jump-statement}\br
    \terminal{break ;}\br
    \terminal{continue ;}\br
    \terminal{return} expression\opt \terminal{;}\br
    \terminal{return} braced-init-list \terminal{;}\br
    \added{coroutine-return-statement} \br
    \terminal{goto} identifier \terminal{;}
  \end{bnf}
\end{quote}

Add the underlined text to paragraph 2:

\begin{quote}
\setcounter{Paras}{1}
\pnum
On exit from a scope (however accomplished), objects with automatic storage
duration~(\cxxref{basic.stc.auto}) that have been constructed in that scope are destroyed
in the reverse order of their construction.
\added{
\enternote
A suspension of a coroutine (\ref{expr.await}) is not considered to be an exit from a scope.
\exitnote
}
...
\end{quote}

\setcounter{subsection}{2}
\rSec2[stmt.return]{The \tcode{return} statement}%
\indextext{\idxcode{return}}%
\indextext{function~return|see{\tcode{return}}}%

%Add the underlined text to paragraph 1:
%
%\begin{quote}
%\pnum
%A function returns to its caller by the \tcode{return} statement\added{; that function shall not be a coroutine (\ref{dcl.fct.def.coroutine}).}
%\added{A return statement shall not appear in a coroutine. A coroutine return statement shall be used instead \ref{stmt.return.coroutine}.}
%\added{In this section, function refers to a function that is not a coroutine. The \tcode{return} statement in a coroutine described in section \ref{stmt.return.coroutine}.}
% \added{A \tcode{return} statement shall not appear in a coroutine.}
%\added{Using a \tcode{return} statement in a coroutine makes the program ill-formed.}
%\end{quote}

Add the underlined text to the last sentence of paragraph 2:

\begin{quote}
\setcounter{Paras}{1}
  \pnum ... Flowing off the end of a function \added{that is not a coroutine} is equivalent to a \tcode{return} with no value; this results in
  undefined behavior in a value-returning function.
\end{quote}

%NOTE

%Add a note:
%
%\begin{quote}
%\enternote
%In this section a function refers to non-coroutines only.
%The return statement in coroutines is described in \ref{stmt.return.coroutine}
%\exitnote
%\end{quote}

%Modify paragraphs 1 through 3 as follows.
%
%\begin{quote}
%\pnum
%A function returns to its caller by the \tcode{return} statement.
%\added{A coroutine also returns to its caller 
%when suspended at suspend-resume point.}
%
%\pnum
%\added{In a non-coroutine a}\removed{A} return statement
%with neither an \grammarterm{expression} nor a \grammarterm{braced-init-list}
%can be used only in functions
%that do not return a value, that is, a function with the return type
%\cv\ \tcode{void}, a constructor~(\cxxref{class.ctor}), or a
%destructor~(\cxxref{class.dtor}).
%\indextext{\idxcode{return}!constructor~and}%
%\indextext{\idxcode{return}!constructor~and}%
%\added{In a coroutine a  return statement
%	with neither an \grammarterm{expression} nor a \grammarterm{braced-init-list}
%	can be used only in functions
%	with eventual return type \tcode{void}.}
%A return statement with an expression of non-void type can be used only
%in \added{non-coroutine} functions returning a value
%\added{or coroutines returning an eventual value}; the value of the expression is returned
%to the caller of the function.
%\indextext{conversion!return~type}%
%The value of the expression is implicitly converted to the return type of the
%function in which it appears. A return statement can involve the
%construction and copy or move of a temporary object~(\cxxref{class.temporary}).
%\enternote
%A copy or move operation associated with a return statement may be elided or
%considered as an rvalue for the purpose of overload resolution in
%selecting a constructor~(\cxxref{class.copy}).
%\exitnote A return statement with a \grammarterm{braced-init-list} initializes the object or reference to be returned from the function by copy-list-initialization~(\cxxref{dcl.init.list}) from the specified initializer list. \enterexample
%
%\begin{codeblock}
%	std::pair<std::string,int> f(const char* p, int x) {
%		return {p,x};
%	}
%\end{codeblock}
%\exitexample
%
%Flowing off the end of a function is equivalent to a \tcode{return} with
%no value; this results in undefined behavior in a value-returning
%\added{non-coroutine} function \added{or in an eventual-value-returning coroutine}.
%
%\pnum
%A return statement with an expression of type \tcode{void}
%can be used only in \added{non-coroutine} functions with a return type of
%\cvqual{cv} \tcode{void} \added{or coroutines with eventual return type of \tcode{void}}; 
%the expression is evaluated just before the function
%returns to its caller.
%\end{quote}
%
%%Add underlined text to paragraph 1:
%%\pnum
%%A function returns to its caller by the \tcode{return} statement
%%\added{or by reaching suspend-resume-point}.
%
%Add paragraph 4.
%
%\begin{quote}
%\setcounter{Paras}{3}
%\pnum 
%In a coroutine return statement is replaced with
%a call to __pr.set_result, where _p
%\end{quote}

\rSec3[stmt.return.coroutine]{The \tcode{co_return} statement}%

Add this subclause to \ref{stmt.jump}.

\begin{quote}
%\enternote
%In this section function refers to coroutine only.
%The return statement in non-coroutines is described in \ref{stmt.return}
%\exitnote

\begin{bnf}
	\nontermdef{coroutine-return-statement}\br
	\terminal{co_return} expression\opt \terminal{;}\br
	\terminal{co_return} braced-init-list\terminal{;}
\end{bnf}

\pnum
A coroutine returns to its caller or resumer (\ref{dcl.fct.def.coroutine}) by the \tcode{co_return} statement
or when suspended (\ref{expr.await}). A coroutine shall not return to its caller or resumer by a \tcode{return} statement (\ref{stmt.return}).

\pnum
The \grammarterm{expression} or \grammarterm{braced-init-list} of a \tcode{co_return} statement is called its operand.
Let $p$ be an lvalue naming the coroutine promise object (\ref{dcl.fct.def.coroutine}) and $P$ be the type of that object, 
then a \tcode{co_return} statement is equivalent to:

%Let $P$ be the coroutine promise type (\ref{dcl.fct.def.coroutine}) and $p$ be an lvalue naming the coroutine promise object (\ref{dcl.fct.def.coroutine}), 
 
%the the \grammarterm{unqualified-id}{s}
%\tcode{return_void} and \tcode{return_value} are looked up in the scope of class $P$. 
%If both are found, the program is ill-formed. If none are found, the coroutine does not have an eventual return type.
%If \tcode{return_void} was found, the coroutine has an eventual return type of \tcode{void}.
%If \tcode{return_value} was found, the coroutine has non-void eventual return type.
%\pnum

\begin{codeblock}
  { @$S$@; goto @$final{\_}suspend$@; }
\end{codeblock}

where $final$\tcode{\_}$suspend$ is as defined in \ref{dcl.fct.def.coroutine} and $S$ is defined as follows:

\begin{itemize}
  \item $S$ is  $p$\tcode{.return_value(}\grammarterm{braced-init-list}{}\tcode{)}, if the operand is a \grammarterm{braced-init-list};
  \item $S$ is  $p$\tcode{.return_value(}\grammarterm{expression}{}\tcode{)}, if the operand is an expression of non-\tcode{void} type;
  \item $S$ is \tcode{\{}{ }\grammarterm{expression}\opt \tcode{;} $p$\tcode{.return_void()}\tcode{;{ }\}}, otherwise;
\end{itemize}
$S$ shall be a prvalue of type \tcode{void}.

\pnum 
If $p$\tcode{.return_void()} is a valid expression, flowing off the end of a coroutine is equivalent to a \tcode{co_return} with no operand; otherwise flowing off the end of a coroutine results in undefined behavior.
%If none are found, the coroutine does not have an eventual return type.
%  $expression$ is treated as an xvalue;

% this is fishy ^^^^ 

%\enternote
%If coroutine instance lifetime is controlled 
%by a RAII object, it is expected that \tcode{_Pr.final_suspend()} returns true
%and coroutine state will be destroyed when
%destructor of an owner object runs.
%
%For detached tasks, where lifetime of the coroutine ends
%when the task completes, \tcode{_Pr.final_suspend()} would return false.
%\exitnote

\end{quote}

%\rSec2[stmt.yield]{The \tcode{yield} statement}%
%
%Add this section to \ref{stmt.jump}.
%
%\begin{quote}
%
%Let \textit{yielded value} be the operand of the \tcode{co_yield} statement and \textit{p} be the promise object of the enclosing coroutine.
%If the result type of \tcode{\textit{p}.yield_value(\textit{yielded-value})} is of type \cvvoid, then the \tcode{co_yield} statement is equivalent to:
%
%\begin{codeblock}
%  @\textit{p}@.yield_value(@\textit{yielded-value}@);
%  @\textit{suspend-resume-point}
%\end{codeblock}
%
%otherwise, it is equivalent to:
%
%\begin{codeblock}
%  if (@\textit{p}@.yield_value(@\textit{yielded-value}@)) {
%    @\textit{suspend-resume-point}@
%  }
%  
%\end{codeblock}

%\pnum
%A \tcode{co_yield} statement may only appear if a \tcode{yield_value} member
%function is defined in the promise type of the enclosing coroutine.

%NOTE: not pretty, can we remove the duplication
%NOTE: make if first
%
%\pnum
%\enternote
%A promise object may have more than one overload of a \tcode{yield_value}.
%
%\enterexample
%\begin{codeblock}
%  recursive_generator<int> flatten(node* n)
%  {
%    if (n == nullptr)
%      return;
%    
%    co_yield flatten(n->left);
%    co_yield n->value;
%    co_yield flatten(n->right);
%  }
%\end{codeblock}
%
%The promise for the \tcode{flatten} function should contain overloads that can accept a value of type \tcode{int} and a value of type \tcode{recursive_generator<int>}.
%In the former case, yielding a value is unconditional. In the latter case, the nested generator may produce an empty sequence of values and thus suspension at the yield point no yielding occurs and  \tcode{yield_value} should return \tcode{false}. 
%\exitexample
%\exitnote

%\end{quote}