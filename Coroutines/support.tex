
\setcounter{chapter}{17}
\rSec0[language.support]{Language support library}

\rSec1[support.general]{General}

Add a row to Table~\ref{tab:lang.sup.lib.summary} for \tcode{<experimental/resumable>}


\begin{libsumtab}{Language support library summary}{tab:lang.sup.lib.summary}
	\cxxref{support.types}       & Types                     &   \tcode{<cstddef>}   \\ \rowsep
	&                           &   \tcode{<limits>}    \\
	\cxxref{support.limits}      & Implementation properties &   \tcode{<climits>}   \\
	&                           &   \tcode{<cfloat>}    \\ \rowsep
	\cxxref{cstdint}             & Integer types             & \tcode{<cstdint>}     \\ \rowsep
	\cxxref{support.start.term}  & Start and termination     &   \tcode{<cstdlib>}   \\ \rowsep
	\cxxref{support.dynamic}     & Dynamic memory management &   \tcode{<new>}       \\ \rowsep
	\cxxref{support.rtti}        & Type identification       &   \tcode{<typeinfo>}  \\ \rowsep
	\cxxref{support.exception}   & Exception handling        &   \tcode{<exception>} \\ \rowsep
	\cxxref{support.initlist}    & Initializer lists & \tcode{<initializer_list>}    \\ \rowsep
	\added{\ref{support.resumable}} 
  & \added{Resumable functions support} 
  & \added{\tcode{<experimental/resumable>}}    \\ \rowsep
	&                           &   \tcode{<csignal>}   \\
	&                           &   \tcode{<csetjmp>}   \\
	&                           &   \tcode{<cstdalign>} \\
	\cxxref{support.runtime}     & Other runtime support     &   \tcode{<cstdarg>}   \\
	&                           &   \tcode{<cstdbool>}  \\
	&                           &   \tcode{<cstdlib>}   \\
	&                           &   \tcode{<ctime>}     \\
\end{libsumtab}


Add section \ref{support.resumable}

\setcounter{section}{10}
\rSec1[support.resumable]{Resumable functions support library}

\pnum
The header
\tcode{<experimental/resumable>}
defines several types providing compile and run-time support for resumable functions in a \Cpp program.

\synopsis{Header \tcode{<experimental/resumable>} synopsis}

\indextext{\idxhdr{experimental/resumable}}%
\indexlibrary{\idxhdr{experimental/resumable}}%
\begin{codeblock}
namespace std {
namespace experimental {
  // \ref{resumable.traits} resumable traits
  template <typename R, typename... ArgTypes>
    class resumable_traits;
	
  // \ref{resumable.handle} resumable handle
  template <typename Promise = void>
    class resumable_handle;		
	
  bool operator == (resumable_handle<> x, resumable_handle<> y) noexcept;
  bool operator < (resumable_handle<> x, resumable_handle<> y) noexcept;			
  bool operator != (resumable_handle<> x, resumable_handle<> y) noexcept;
  bool operator <= (resumable_handle<> x, resumable_handle<> y) noexcept;			
  bool operator >= (resumable_handle<> x, resumable_handle<> y) noexcept;
  bool operator > (resumable_handle<> x, resumable_handle<> y) noexcept;			
}
}
\end{codeblock}

\rSec2[resumable.traits]{resumable traits}
\pnum
This subclause defines requirements on classes representing
\term{resumable traits},
and defines a primary struct template
\tcode{resumable_traits<R,Args...>}
that satisfies those requirements.

%\pnum
%Resumable functions need a set of related types and functions
%to complete the definition of their semantics.
%These types and functions are provided as a set of member types or typedefs
%and functions in the instantiation of struct template
%\tcode{resumable_traits}. This subclause defines the semantics of these
%members.

\pnum
The \tcode{resumable_traits} may be specialized by the user 
to customize semantics of resumable functions.

\rSec3[resumable.traits.requirements]{Resumable traits requirements}
\pnum
In Table~\ref{tab:resumable.traits.requirements}, X denotes 
a trait class instantiated as described in \ref{dcl.fct.def.resumable};
$a_1$, $a_2$, ... $a_n$ denote parameters passed to a resumable function. If it is a member function, than $a_1$ denotes implicit \tcode{this} parameter.

\begin{concepttable}{\tcode{resumable_traits} requirements}{tab:resumable.traits.requirements}
  {p{2.4in}p{3.35in}}
  \topline
  Expression          &   Behavior \\ \capsep
  \tcode{X::promise_type}     &
  \tcode{X::promise_type} must be a type satisfying resumable promise requirements (\ref{resumable.promise}) 
%  that completes the definition of resumable function semantics.
  \\ \rowsep
  \tcode{X::get_allocator($a_1$, $a_2$, ... $a_n$)}        &
  \textit{(optional)} Given a set of arguments passed to a resumable function,
  returns an allocator (\cxxref{allocator.requirements}) that implementation can use
  to dynamically allocate memory for objects with automatic
  storage duration in a resumable function if required.
  If \tcode{get_allocator} is not present, 
  the implementation shall use \tcode{std::allocator<char>}.
  \\ \rowsep
  \tcode{X::construct_promise(X::promise_type* $p$)}        &
  \textit{(optional)} Constructs a \tcode{X::promise_type} at the given
  address $p$ (\cxxref{expr.new}). 
  If \tcode{construct_promise} is not present, 
  the implementation shall in-place construct the promise_type (use 
  \tcode{::new (p) X::promise_type()}).
  \\ \rowsep
  \tcode{X::destruct_promise(X::promise_type* $p$)}        &
  \textit{(optional)} Destruct a \tcode{X::promise_type} at the given
  address $p$.
  If \tcode{destruct_promise} is not present, 
  the implementation shall use \tcode{p->~X::promise_type()}.
\\ \rowsep
  \tcode{X::get_return_object_on_allocation_failure()}     &
\textit{(optional)} If present, it is assumed that an
allocator's \tcode{allocate} function will violate the standard requirements and return \tcode{nullptr} in case of an
allocation failure. If a resumable function requires dynamic allocation, it must check if an \tcode{allocate} returns nullptr, and if so it shall use the expression \tcode{X::get_return_object_on_allocation_failure()} to construct the return value and return back to the caller.
  \\ 
\end{concepttable}

\rSec3[resumable.traits.primary]{Struct template \tcode{resumable_traits}}
\pnum The header \tcode{<resumable>} shall define
primary struct template \tcode{resumable_traits} as follows:

%\pnum The requirements for the members are given in clause \ref{resumable.traits.requirements}.
\indexlibrary{\idxcode{resumable_traits}}%
\begin{codeblock}
namespace std {
namespace experimental {
  template <typename R, typename... Args>
  struct resumable_traits {
    using promise_type = typename R::promise_type;
  };
} // namespace experimental
} // namespace std
\end{codeblock}

\rSec2[resumable.handle]{Struct template \tcode{resumable_handle}}

\indexlibrary{\idxcode{resumable_handle}}%
\begin{codeblock}
namespace std {
  namespace experimental {
    template <>
    struct resumable_handle<void>
    {
      // \ref{resumable.handle.con} construct/reset
      resumable_handle() noexcept;		
      resumable_handle(std::nullptr_t) noexcept;
      resumable_handle& operator=(nullptr_t) noexcept;
      
      // \ref{resumable.handle.export} export/import
      static resumable_handle from_address(void* addr) noexcept;		
      void* to_address() const noexcept;
      
      // \ref{resumable.handle.capacity} capacity
      explicit operator bool() const noexcept;
      
      // \ref{resumable.handle.resumption} resumption
      void operator()() const;
      void resume() const;	
      void destroy() const;
      
      // \ref{resumable.handle.completion} completion check
      bool done() const noexcept; 
    };
    
    template <typename Promise>
    struct resumable_handle : resumable_handle<>
    {
      // \ref{resumable.handle.con} construct/reset
      using resumable_handle<>::resumable_handle;
      resumable_handle(Promise*) noexcept;		
      resumable_handle& operator=(nullptr_t) noexcept;
      
      // \ref{resumable.handle.prom} promise access
      Promise& promise() noexcept;		
      Promise const& promise() const noexcept;
    };
  }
}
\end{codeblock}

\pnum
The struct template \tcode{resumable_handle}
can be used to refer to a suspended or executing resumable function.
Such function is called a \textit{target} of \tcode{resumable_handle}.
A default constructed \tcode{resumable_handle} object has no target.


\rSec3[resumable.handle.con]{\tcode{resumable_handle} construct/reset}
\begin{itemdecl}
  resumable_handle() noexcept;		
  resumable_handle(std::nullptr_t) noexcept;
\end{itemdecl}
\begin{itemdescr}
  \pnum\postconditions \tcode{!*this}.
\end{itemdescr}

\begin{itemdecl}
  resumable_handle(Promise* p) noexcept;	
\end{itemdecl}
\begin{itemdescr}
  \pnum
  \precondition \tcode{p} points to a promise object of a resumable function.
  
	\pnum
  \postconditions \tcode{!*this} and \tcode{addressof(this->promise()) == p}.
\end{itemdescr}

\begin{itemdecl}
  resumable_handle& operator=(nullptr_t) noexcept;
\end{itemdecl}
\begin{itemdescr}
	\pnum\postconditions \tcode{this->operator bool() == false}.
  
  \pnum\returns \tcode{*this}.
\end{itemdescr}

\rSec3[resumable.handle.export]{\tcode{resumable_handle} export/import}
\begin{itemdecl}
  static resumable_handle from_address(void* addr) noexcept;		
  void* to_address() const noexcept;
\end{itemdecl}

\begin{itemdescr}
  \pnum
  \postconditions \tcode{resumable_traits<>::from_address(this->to_address()) == *this}.
\end{itemdescr}

\rSec3[resumable.handle.capacity]{\tcode{resumable_handle} capacity}
\begin{itemdecl}
  explicit operator bool() const noexcept;
\end{itemdecl}

\begin{itemdescr}
  \pnum
  \returns \tcode{true} if \tcode{*this} has a target, otherwise \tcode{false}.
\end{itemdescr}

\rSec3[resumable.handle.resumption]{\tcode{resumable_handle} resumption}
\begin{itemdecl}
  void operator()() const;
  void resume() const;	
\end{itemdecl}
\begin{itemdescr}
  \pnum
  \precondition *this refers to a suspended resumable function
  
  \pnum
  \effects resumes the execution of a target function. If the function was suspended
  at the final suspend point, std::terminate is called (\cxxref{except.terminate}).
\end{itemdescr}

\begin{itemdecl}
  void destroy() const;
\end{itemdecl}
\begin{itemdescr}
  \pnum
  \precondition *this refers to a suspended resumable function
  
  \pnum
  \effects objects with automatic storage duration that are in scope
  at the suspend point are destroyed in the reverse order of the construction. If resumable function required dynamic allocation
  for the objects with automatic storage duration, the memory
  is freed.
\end{itemdescr}

\rSec3[resumable.handle.completion]{\tcode{resumable_handle} completion check}
\begin{itemdecl}
  bool done() const noexcept; 
\end{itemdecl}
\begin{itemdescr}
  \pnum
  \precondition *this refers to a suspended resumable function
  
  \pnum
  \returns \tcode{true} if target function is suspended
  at final suspend point, otherwise \tcode{false}.
\end{itemdescr}

\rSec3[resumable.handle.prom]{\tcode{resumable_handle} promise access}
\begin{itemdecl}
  Promise& promise() noexcept;		
  Promise const& promise() const noexcept;
\end{itemdecl}

\begin{itemdescr}
  \pnum
  \precondition *this refers to a resumable function
  
  \pnum
  \returns a reference to a promise of the target function.
\end{itemdescr}


\rSec2[resumable.promise]{Resumable promise requirements}

\pnum
A user supplies the definition of the resumable promise to implement 
desired high-level semantics associated with a resumable functions
discovered via instantiation of struct template \tcode{resumable_traits}.
The following tables describe the requirements on
resumable promise types.

%\pnum
%The template struct \tcode{allocator_traits}~(\ref{allocator.traits}) supplies
%a uniform interface to all allocator types.
%Table~\ref{tab:desc.var.def} describes the types manipulated
%through allocators. Table~\ref{tab:utilities.allocator.requirements}
%describes the requirements on allocator types
%and thus on types used to instantiate \tcode{allocator_traits}. A requirement
%is optional if the last column of
%Table~\ref{tab:utilities.allocator.requirements} specifies a default for a
%given expression. Within the standard library \tcode{allocator_traits}
%template, an optional requirement that is not supplied by an allocator is
%replaced by the specified default expression. A user specialization of
%\tcode{allocator_traits} may provide different defaults and may provide
%defaults for different requirements than the primary template. Within
%Tables~\ref{tab:desc.var.def} and~\ref{tab:utilities.allocator.requirements},
%the use of \tcode{move} and \tcode{forward} always refers to \tcode{std::move}
%and \tcode{std::forward}, respectively.

\begin{libreqtab2}
	{Descriptive variable definitions}
	{tab:desc.var.def}
	\\ \topline
	\lhdr{Variable} &   \rhdr{Definition}   \\  \capsep
	\endfirsthead
	\continuedcaption\\
	\hline
	\lhdr{Variable} &   \rhdr{Definition}   \\  \capsep
	\endhead
	\tcode{P}    &   a resumable promise type       \\ \rowsep
	\tcode{p}       &   a value of type \tcode{P} \\ \rowsep
	\tcode{e}       &   a value of \tcode{std::exception_ptr} type   \\ \rowsep
	\tcode{h}       &   a value of \tcode{std::experimental::resumable_handle<P>} type    \\ \rowsep
	\tcode{v}      &   an \grammarterm{expression} or \grammarterm{braced-init-list}   \\ \rowsep
\end{libreqtab2}

\indextext{requirements!\idxcode{ResumablePromise}}%
\begin{concepttable}{\tcode{ResumablePromise} requirements}{ResumablePromise}
	{p{1.6in}p{4.15in}}
	\topline
	Expression          &   Note \\ \capsep
	\tcode{P\{\}}     &   Construct an object of type \tcode{P}\\ \rowsep
	\tcode{p.get_return_object()}        &
\tcode{get_return_object} is invoked by the resumable function to construct the
return object prior to reaching the first suspend-resume point,
a \tcode{return} statement or flowing off the end of the function.
	\\ \rowsep
	\tcode{p.return_value(v)}     &  
If present, an enclosing resumable function supports an
eventual value of a type that \tcode{v} can be converted to. \tcode{return_value} is invoked by
a resumable function when 
a return statement
with an \grammarterm{expression} 
or a \grammarterm{braced-init-list}
is encountered in a resumable function.

If a promise type does not satisfy this requirement, the presence of 
a return statement
with an \grammarterm{expression} 
or a \grammarterm{braced-init-list}
statement in the body results in a compile time error.
	\\ \rowsep
	\tcode{p.return_void()}     &   
If present, an enclosing resumable function supports an eventual value of type \tcode{void}. \tcode{return_void} is invoked by
a resumable function when 
a return statement
without an \grammarterm{expression} 
nor a \grammarterm{braced-init-list}
is encountered in a resumable function or
control flows to the end of the function.
A promise type shall not define both \tcode{return_void} and \tcode{return_value} member functions.
	\\ \rowsep
	\tcode{p.set_exception(e)} & 
\tcode{set_exception} is invoked by a resumable function when an
unhandled exception occurs within a \grammarterm{function-body} of the resumable
function.
If promise does not provide \tcode{set_exception}, an unhandled exception
will propagate from a resumable functions normally.
\\ \rowsep
	\tcode{p.yield_value(v)}     &   
Must be present for the enclosing resumable function to support \tcode{yield} statement.
	\\ \rowsep
	\tcode{p.initial_suspend()}     &
if \tcode{p.initial_suspend()} evaluates to \tcode{true}, resumable function will suspend at \textit{initial suspend point} (\ref{dcl.fct.def.resumable}).
	   \\ \rowsep
	\tcode{p.final_suspend()}     &  
if \tcode{p.final_suspend()} evaluates to \tcode{true}, resumable function will suspend at \textit{final suspend point} (\ref{dcl.fct.def.resumable}).
\\ 
\end{concepttable}

\enterexample
This example illustrates full implementation
of a promise type for a simple generator.
\begin{codeblock}
  #include <iostream>
  #include <experimental/resumable>
  
  struct generator {
    struct promise_type {
      int current_value;
      auto get_return_object() { return generator{this}; }
      auto initial_suspend() { return true; }
      auto final_suspend() { return true; }
      void yield_value(int value) { current_value = value; }
    };
    
    bool move_next() {
      coro.resume();
      return !coro.done();
    }
    
    int current_value() { return coro.promise().current_value; }
    
    ~generator() { coro.destroy(); }
  private:
    explicit generator(promise_type* myPromise) : coro(myPromise) 
    {
    }
    std::experimental::resumable_handle<promise_type> coro;
  };
  
  generator f() {
    yield 1;
    yield 2;
  } 
  
  int main() {
    auto g = f();
    while (g.move_next()) std::cout << g.current_value() << std::endl;
  }
  
\end{codeblock}
\exitexample