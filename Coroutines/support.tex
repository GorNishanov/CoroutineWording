
\setcounter{chapter}{17}
\rSec0[language.support]{Language support library}

\rSec1[support.general]{General}

Add a row to Table~\ref{tab:lang.sup.lib.summary} for \tcode{<experimental/coroutine>}


\begin{libsumtab}{Language support library summary}{tab:lang.sup.lib.summary}
	\cxxref{support.types}       & Types                     &   \tcode{<cstddef>}   \\ \rowsep
	&                           &   \tcode{<limits>}    \\
	\cxxref{support.limits}      & Implementation properties &   \tcode{<climits>}   \\
	&                           &   \tcode{<cfloat>}    \\ \rowsep
	\cxxref{cstdint}             & Integer types             & \tcode{<cstdint>}     \\ \rowsep
	\cxxref{support.start.term}  & Start and termination     &   \tcode{<cstdlib>}   \\ \rowsep
	\cxxref{support.dynamic}     & Dynamic memory management &   \tcode{<new>}       \\ \rowsep
	\cxxref{support.rtti}        & Type identification       &   \tcode{<typeinfo>}  \\ \rowsep
	\cxxref{support.exception}   & Exception handling        &   \tcode{<exception>} \\ \rowsep
	\cxxref{support.initlist}    & Initializer lists & \tcode{<initializer_list>}    \\ \rowsep
	\added{\ref{support.coroutine}} 
  & \added{Coroutines support} 
  & \added{\tcode{<experimental/coroutine>}}    \\ \rowsep
	&                           &   \tcode{<csignal>}   \\
	&                           &   \tcode{<csetjmp>}   \\
	&                           &   \tcode{<cstdalign>} \\
	\cxxref{support.runtime}     & Other runtime support     &   \tcode{<cstdarg>}   \\
	&                           &   \tcode{<cstdbool>}  \\
	&                           &   \tcode{<cstdlib>}   \\
	&                           &   \tcode{<ctime>}     \\
\end{libsumtab}


Add section \ref{support.coroutine}

\setcounter{section}{10}
\rSec1[support.coroutine]{Coroutines support library}

\pnum
The header
\tcode{<experimental/coroutine>}
defines several types providing compile and run-time support for coroutines in a \Cpp program.

\synopsis{Header \tcode{<experimental/coroutine>} synopsis}

\indextext{\idxhdr{experimental/coroutine}}%
\indexlibrary{\idxhdr{experimental/coroutine}}%
\begin{codeblock}
namespace std {
namespace experimental {
  // \ref{coroutine.traits} coroutine traits
  template <typename R, typename... ArgTypes>
    class coroutine_traits;
	
  // \ref{coroutine.handle} coroutine handle
  template <typename Promise = void>
    class coroutine_handle;		
	
  bool operator == (coroutine_handle<> x, coroutine_handle<> y) noexcept;
  bool operator < (coroutine_handle<> x, coroutine_handle<> y) noexcept;			
  bool operator != (coroutine_handle<> x, coroutine_handle<> y) noexcept;
  bool operator <= (coroutine_handle<> x, coroutine_handle<> y) noexcept;			
  bool operator >= (coroutine_handle<> x, coroutine_handle<> y) noexcept;
  bool operator > (coroutine_handle<> x, coroutine_handle<> y) noexcept;			
}
}
\end{codeblock}

\rSec2[coroutine.traits]{coroutine traits}
\pnum
This subclause defines requirements on classes representing
\term{coroutine traits},
and defines a primary struct template
\tcode{coroutine_traits<R,Args...>}
that satisfies those requirements.

%\pnum
%Coroutines need a set of related types and functions
%to complete the definition of their semantics.
%These types and functions are provided as a set of member types or typedefs
%and functions in the instantiation of struct template
%\tcode{coroutine_traits}. This subclause defines the semantics of these
%members.

\pnum
The \tcode{coroutine_traits} may be specialized by the user 
to customize semantics of coroutines.

\rSec3[coroutine.traits.requirements]{Coroutine traits requirements}
\pnum
In Table~\ref{tab:coroutine.traits.requirements}, X denotes 
a trait class instantiated as described in \ref{dcl.fct.def.coroutine};
$a_1$, $a_2$, ... $a_n$ denote parameters passed to a coroutine. If it is a member function, than $a_1$ denotes implicit \tcode{this} parameter.

\begin{concepttable}{\tcode{coroutine_traits} requirements}{tab:coroutine.traits.requirements}
  {p{1.6in}p{4.15in}}
  \topline
  Expression          &   Behavior \\ \capsep
  \tcode{X::promise_type}     &
  \tcode{X::promise_type} must be a type satisfying coroutine promise requirements (\ref{coroutine.promise}) 
%  that completes the definition of coroutine semantics.
  \\ \rowsep
  \tcode{X::get_allocator($a_1$, $a_2$, ... $a_n$)}        &
  \textit{(optional)} Given a set of arguments passed to a coroutine,
  returns an allocator (\cxxref{allocator.requirements}) that implementation can use
  to dynamically allocate memory for objects with automatic
  storage duration in a coroutine if required.
  If \tcode{get_allocator} is not present, 
  implementation shall use \tcode{std::allocator<char>}.
\\ \rowsep
  \tcode{X::get_return_object_on_allocation_failure()}     &
\textit{(optional)} If present, it is assumed that an
allocator's \tcode{allocate} function will violate the standard requirements and return \tcode{nullptr} in case of an
allocation failure. If a coroutine requires dynamic allocation, it must check if an \tcode{allocate} returns nullptr, and if so it shall use the expression \tcode{X::get_return_object_on_allocation_failure()} to construct the return value and return back to the caller.
  \\ 
\end{concepttable}

\rSec3[coroutine.traits.primary]{Struct template \tcode{coroutine_traits}}
\pnum The header \tcode{<coroutine>} shall define
primary struct template \tcode{coroutine_traits} as follows:

%\pnum The requirements for the members are given in clause \ref{coroutine.traits.requirements}.
\indexlibrary{\idxcode{coroutine_traits}}%
\begin{codeblock}
namespace std {
namespace experimental {
  template <typename R, typename... Args>
  struct coroutine_traits {
    using promise_type = typename R::promise_type;
  };
} // namespace experimental
} // namespace std
\end{codeblock}

\rSec2[coroutine.handle]{Struct template \tcode{coroutine_handle}}

\indexlibrary{\idxcode{coroutine_handle}}%
\begin{codeblock}
namespace std {
  namespace experimental {
    template <>
    struct coroutine_handle<void>
    {
      // \ref{coroutine.handle.con} construct/reset
      coroutine_handle() noexcept;		
      coroutine_handle(std::nullptr_t) noexcept;
      coroutine_handle& operator=(nullptr_t) noexcept;
      
      // \ref{coroutine.handle.export} export/import
      static coroutine_handle from_address(void* addr) noexcept;		
      void* to_address() const noexcept;
      
      // \ref{coroutine.handle.capacity} capacity
      explicit operator bool() const noexcept;
      
      // \ref{coroutine.handle.resumption} resumption
      void operator()() const;
      void resume() const;	
      void destroy() const;
      
      // \ref{coroutine.handle.completion} completion check
      bool done() const noexcept; 
    };
    
    template <typename Promise>
    struct coroutine_handle : coroutine_handle<>
    {
      // \ref{coroutine.handle.con} construct/reset
      using coroutine_handle<>::coroutine_handle;
      coroutine_handle(Promise*) noexcept;		
      coroutine_handle& operator=(nullptr_t) noexcept;
      
      // \ref{coroutine.handle.prom} promise access
      Promise& promise() noexcept;		
      Promise const& promise() const noexcept;
    };
  }
}
\end{codeblock}

\pnum
The struct template \tcode{coroutine_handle}
can be used to refer to a suspended or executing coroutine.
Such function is called a \textit{target} of \tcode{coroutine_handle}.
A default constructed \tcode{coroutine_handle} object has no target.


\rSec3[coroutine.handle.con]{\tcode{coroutine_handle} construct/reset}
\begin{itemdecl}
  coroutine_handle() noexcept;		
  coroutine_handle(std::nullptr_t) noexcept;
\end{itemdecl}
\begin{itemdescr}
  \pnum\postconditions \tcode{!*this}.
\end{itemdescr}

\begin{itemdecl}
  coroutine_handle(Promise* p) noexcept;	
\end{itemdecl}
\begin{itemdescr}
  \pnum
  \precondition \tcode{p} points to a promise object of a coroutine.
  
	\pnum
  \postconditions \tcode{!*this} and \tcode{addressof(this->promise()) == p}.
\end{itemdescr}

\begin{itemdecl}
  coroutine_handle& operator=(nullptr_t) noexcept;
\end{itemdecl}
\begin{itemdescr}
	\pnum\postconditions \tcode{this->operator bool() == false}.
  
  \pnum\returns \tcode{*this}.
\end{itemdescr}

\rSec3[coroutine.handle.export]{\tcode{coroutine_handle} export/import}
\begin{itemdecl}
  static coroutine_handle from_address(void* addr) noexcept;		
  void* to_address() const noexcept;
\end{itemdecl}

\begin{itemdescr}
  \pnum
  \postconditions \tcode{coroutine_traits<>::from_address(this->to_address()) == *this}.
\end{itemdescr}

\rSec3[coroutine.handle.capacity]{\tcode{coroutine_handle} capacity}
\begin{itemdecl}
  explicit operator bool() const noexcept;
\end{itemdecl}

\begin{itemdescr}
  \pnum
  \returns \tcode{true} if \tcode{*this} has a target, otherwise \tcode{false}.
\end{itemdescr}

\rSec3[coroutine.handle.resumption]{\tcode{coroutine_handle} resumption}
\begin{itemdecl}
  void operator()() const;
  void resume() const;	
\end{itemdecl}
\begin{itemdescr}
  \pnum
  \precondition *this refers to a suspended coroutine
  
  \pnum
  \effects resumes the execution of a target function. If the function was suspended
  at the final suspend point, std::terminate is called (\cxxref{except.terminate}).
\end{itemdescr}

\begin{itemdecl}
  void destroy() const;
\end{itemdecl}
\begin{itemdescr}
  \pnum
  \precondition *this refers to a suspended coroutine
  
  \pnum
  \effects objects with automatic storage duration that are in scope
  at the suspend point are destroyed in the reverse order of the construction. If coroutine required dynamic allocation
  for the objects with automatic storage duration, the memory
  is freed.
\end{itemdescr}

\rSec3[coroutine.handle.completion]{\tcode{coroutine_handle} completion check}
\begin{itemdecl}
  bool done() const noexcept; 
\end{itemdecl}
\begin{itemdescr}
  \pnum
  \precondition *this refers to a suspended coroutine
  
  \pnum
  \returns \tcode{true} if target function is suspended
  at final suspend point, otherwise \tcode{false}.
\end{itemdescr}

\rSec3[coroutine.handle.prom]{\tcode{coroutine_handle} promise access}
\begin{itemdecl}
  Promise& promise() noexcept;		
  Promise const& promise() const noexcept;
\end{itemdecl}

\begin{itemdescr}
  \pnum
  \precondition *this refers to a coroutine
  
  \pnum
  \returns a reference to a promise of the target function.
\end{itemdescr}


\rSec2[coroutine.promise]{Coroutine promise requirements}

\pnum
A user supplies the definition of the coroutine promise to implement 
desired high-level semantics associated with a coroutines
discovered via instantiation of struct template \tcode{coroutine_traits}.
The following tables describe the requirements on
coroutine promise types.

%\pnum
%The template struct \tcode{allocator_traits}~(\ref{allocator.traits}) supplies
%a uniform interface to all allocator types.
%Table~\ref{tab:desc.var.def} describes the types manipulated
%through allocators. Table~\ref{tab:utilities.allocator.requirements}
%describes the requirements on allocator types
%and thus on types used to instantiate \tcode{allocator_traits}. A requirement
%is optional if the last column of
%Table~\ref{tab:utilities.allocator.requirements} specifies a default for a
%given expression. Within the standard library \tcode{allocator_traits}
%template, an optional requirement that is not supplied by an allocator is
%replaced by the specified default expression. A user specialization of
%\tcode{allocator_traits} may provide different defaults and may provide
%defaults for different requirements than the primary template. Within
%Tables~\ref{tab:desc.var.def} and~\ref{tab:utilities.allocator.requirements},
%the use of \tcode{move} and \tcode{forward} always refers to \tcode{std::move}
%and \tcode{std::forward}, respectively.

\begin{libreqtab2}
	{Descriptive variable definitions}
	{tab:desc.var.def}
	\\ \topline
	\lhdr{Variable} &   \rhdr{Definition}   \\  \capsep
	\endfirsthead
	\continuedcaption\\
	\hline
	\lhdr{Variable} &   \rhdr{Definition}   \\  \capsep
	\endhead
	\tcode{P}    &   a coroutine promise type       \\ \rowsep
	\tcode{p}       &   a value of type \tcode{P} \\ \rowsep
	\tcode{e}       &   a value of \tcode{std::exception_ptr} type   \\ \rowsep
	\tcode{h}       &   a value of \tcode{std::experimental::coroutine_handle<P>} type    \\ \rowsep
	\tcode{v}      &   an \grammarterm{expression} or \grammarterm{braced-init-list}   \\ \rowsep
\end{libreqtab2}

\indextext{requirements!\idxcode{CoroutinePromise}}%
\begin{concepttable}{\tcode{CoroutinePromise} requirements}{CoroutinePromise}
	{p{1.6in}p{4.15in}}
	\topline
	Expression          &   Note \\ \capsep
	\tcode{P\{\}}     &   Construct an object of type \tcode{P}\\ \rowsep
	\tcode{p.get_return_object()}        &
The \tcode{get_return_object} is invoked by the coroutine to construct the
return object prior to reaching the first suspend-resume point,
a \tcode{return} statement or flowing off the end of the function.
	\\ \rowsep
	\tcode{p.return_value(v)}     &  
%If present, an enclosing coroutine supports an
%eventual value of a type that \tcode{v} can be converted to. 

Invoked by
a coroutine when 
a \tcode{coreturn} statement
with an \grammarterm{expression} 
or a \grammarterm{braced-init-list}
is encountered in a coroutine (\ref{stmt.return.coroutine}).
%If a promise type does not satisfy this requirement, the presence of 
%a \tcode{coreturn} statement
%with an \grammarterm{expression} 
%or a \grammarterm{braced-init-list}
%statement in the body results in a compile time error.
	\\ \rowsep
	\tcode{p.return_void()}     &   
%If present, an enclosing coroutine supports an eventual value of type \tcode{void}. 
If present, invoked when 
a \tcode{coreturn} statement is encountered as described in (\ref{stmt.return.coroutine}).
A promise type shall not define both \tcode{return_void} and \tcode{return_value} member functions.
	\\ \rowsep
	\tcode{p.set_exception(e)} & 
The \tcode{set_exception} is invoked by a coroutine when an
unhandled exception occurs within a \grammarterm{function-body} of the coroutine
function.
If promise does not provide \tcode{set_exception}, an unhandled exception
will propagate from a coroutines normally.
\\ \rowsep
	\tcode{p.yield_value(v)}     &
  The \tcode{yield_value} is invoked
  when \tcode{coyield} statement is
  encountered in the coroutine. If
  promise does not define \tcode{yield_value}, \tcode{coyield}
  statement may not appear in the coroutine body.
	\\ \rowsep
	\tcode{p.initial_suspend()}     &
if \tcode{p.initial_suspend()} evaluates to \tcode{true}, the coroutine will suspend at \textit{initial suspend point} (\ref{dcl.fct.def.coroutine}).
	   \\ \rowsep
	\tcode{p.final_suspend()}     &  
if \tcode{p.final_suspend()} evaluates to \tcode{true}, the coroutine will suspend at \textit{final suspend point} (\ref{dcl.fct.def.coroutine}).
\\ 
\end{concepttable}

\enterexample
This example illustrates full implementation
of a promise type for a simple generator.
\begin{codeblock}
  #include <iostream>
  #include <experimental/coroutine>
  
  struct generator {
    struct promise_type {
      int current_value;
      auto get_return_object() { return generator{this}; }
      auto initial_suspend() { return true; }
      auto final_suspend() { return true; }
      void yield_value(int value) { current_value = value; }
    };
    
    bool move_next() {
      coro.resume();
      return !coro.done();
    }
    
    int current_value() { return coro.promise().current_value; }
    
    ~generator() { coro.destroy(); }
  private:
    explicit generator(promise_type* myPromise) : coro(myPromise) 
    {
    }
    std::experimental::coroutine_handle<promise_type> coro;
  };
  
  generator f() {
    coyield 1;
    coyield 2;
  } 
  
  int main() {
    auto g = f();
    while (g.move_next()) std::cout << g.current_value() << std::endl;
  }
  
\end{codeblock}
\exitexample