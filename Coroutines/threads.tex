\setcounter{chapter}{29}
\rSec0[thread]{Thread support library}

\setcounter{section}{2}
\rSec1[thread.threads]{Threads}

\setcounter{subsection}{1}
\rSec2[thread.thread.this]{Namespace \tcode{this_thread}}

Rename \tcode{yield}
function to \tcode{yield_execution}.

\begin{quote}
\begin{codeblock}
namespace std {
  namespace this_thread {
    thread::id get_id() noexcept;

    void yield@\added{_execution}@() noexcept;
    template <class Clock, class Duration>
      void sleep_until(const chrono::time_point<Clock, Duration>& abs_time);
    template <class Rep, class Period>
      void sleep_for(const chrono::duration<Rep, Period>& rel_time);
  }
}
\end{codeblock}

\indexlibrary{\idxcode{this_thread}!\idxcode{get_id}}%
\indexlibrary{\idxcode{get_id}!\idxcode{this_thread}}%
\begin{itemdecl}
thread::id this_thread::get_id() noexcept;
\end{itemdecl}

\begin{itemdescr}
\pnum
\returns An object of type \tcode{thread::id} that uniquely identifies the current thread of
execution. No other thread of execution shall have this id and this thread of execution shall
always have this id. The object returned shall not compare equal to a default constructed
\tcode{thread::id}.
\end{itemdescr}

\indexlibrary{\idxcode{this_thread}!\idxcode{yield\added{_execution}}}%
\indexlibrary{\idxcode{yield\added{_execution}}!\idxcode{this_thread}}%
\begin{itemdecl}
void this_thread::yield@\added{_execution}@() noexcept;
\end{itemdecl}

\begin{itemdescr}
\pnum
\effects Offers the implementation the opportunity to reschedule.

\pnum
\sync None.
\end{itemdescr}
\end{quote}